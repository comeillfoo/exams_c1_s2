\documentclass[11pt]{article}

\usepackage[T2A]{fontenc}
\usepackage[utf8x]{inputenc}

\usepackage[russian, english]{babel}

\usepackage{vmargin}
\setpapersize{A4}
\setmarginsrb{1cm}{1cm}{1cm}{1cm}{0pt}{0pt}{0mm}{13mm}

\usepackage{amssymb, amsmath}

\sloppy

\begin{document}
	\section{Неопределенный интеграл и его свойства}
	\paragraph*{Первообразная функции и неопределенный интеграл}
	Основной задачей интегрального исчисления является восстановление функции $F(x)$ по известной производной $f(x)$ (дифференциалу $f(x)dx$) этой функции. Таким образом, интегрирование ~--- это действие, обратное дифференцированию. Результат интегрирования проверяется путем дифференцирования.

	Функция $F(x), x\in X$, называется {\itshape первообразной} для функции $f(x)$ на множестве $X\subset\mathbb{R}$, если она дифференцируема для любого $x\in X$ и $F'(x) = f(x)$ или $dF(x)=f(x)dx$.

	Совокупность $F(x)+C$, где $C$ ~--- произвольная постоянная, всех первообразных функции $f(x)$ на множестве $X$ называется \textit{неопределенным интегралом} и обозначается
	$$
		\int{f(x)dx} = F(x) + C.
	$$
	С геометрической точки зрения неопределенный интеграл представляет собой однопараметрическое семейство кривых $y = F(x) + C$ ($C$ ~--- параметр), обладающих следующим свойством: все касательные к кривым в точках с абсциссой $x = x_0$ параллельны между собой:
	$$
		(F(x) + C)_{x = x_0}' = F'(x) = f(x_0)
	$$
	\paragraph*{Основные свойства неопределенного интеграла}
	\begin{enumerate}
		\item Производная от неопределенного интеграла равна подынтегральной функции; дифференциал от неопределенного интеграла равен подынтегральному выражению:
		\begin{align}
			\left(\int{f(x)dx}\right)' &= f(x) & \text{или} & d\left(\int{f(x)dx}\right) &= f(x)dx.\\
			\int{dF(x)} &= F(x) + C & \text{или} & \int{F'(x)dx} = F(x) + C.
		\end{align}
		\item Постоянный множитель можно выносить за знак интеграла т.е.
		$$
			\int{af(x)dx} = a\cdot\int{f(x)dx}.
		$$
		\item Интеграл от суммы равен сумме интегралов от всех слагаемых:
		$$
			\int{\left(f_1(x)\pm f_2(x)\pm\ldots\pm f_n(x)\right)dx} = \int{f_1(x)dx}\pm\int{f_2(x)dx}\pm\ldots\pm\int{f_n(x)dx}
		$$
		\item Если $F(x)$ ~--- первообразная функции $f(x)$, то
		$$
			\int{f(ax + b)dx} = \frac{1}{a}F(ax + b) + C.
		$$
		\item\textbf{Инвариантность формул интегрирования.} Любая формула интегрирования сохраняет свой вид, если переменную интегрирования заменить любой дифференцируемой функцией этой переменной:
		$$
			\int{f(x)dx} = F(x) + C \Rightarrow \int{f(u)du} = F(u) + C,
		$$
		где $u$ ~--- дифференцируемая функция.
	\end{enumerate}
	\section{Интегрирование по частям и замена переменной}
	\begin{enumerate}
		\item\textbf{Интегрирование подстановкой (заменой переменной)} ~--- состоит в том, что в интеграле $\int{f(x)dx}$ переменную $x$ заменяют функцией от переменной $t$ по формуле $x = \varphi (t)$, откуда $dx = \varphi '(t)dt$, т.е. имеет место формула
		$$
			\int{f(x)dx} = \int{f(\varphi (t))\varphi '(t)dt}.
		$$
		\item\textbf{Интегрирование по частям.} Пусть $u(x)$ и $v(x)$ ~--- две дифференцируемые функции переменной $x$, тогда соотношение
		$$
			\int{udv} = uv - \int{vdu},
		$$
		называется \textit{формулой интегрирования по частям}. Эта формула получается из формулы дифференциала произведения $d(uv) = udv + vdu$ интегрированием обеих частей.

		Приведем некоторые часто встречающиеся типы интегралов, вычисляемых методом интегрирования по частям.
		\begin{enumerate}
			\item Интегралы вида $\int{P(x)e^{kx}dx}$, $\int{P(x)\sin{kx}dx}$, $\int{P(x)\cos{kx}dx}$. (P(x) ~--- многочлен, k ~--- некоторое число).
			Для нахождения этих интегралов за $u$ принимают многочлен.
			\item Интегралы вида $\int{P(x)\ln{x}dx}$, $\int{P(x)\arcsin{x}dx}$, $\int{P(x)\arccos{x}dx}$, $\int{P(x)\arctg{x}dx}$, $\int{P(x)\arcctg{x}dx}$.
			Для нахождения этих интегралов за $u$ принимают множитель, стоящий при многочлене.
		\end{enumerate}
	\end{enumerate}
	\section{Интегрирование рациональных дробей}
	\textit{Рациональной дробью} называется функция, которая может быть представлена в виде отношения двух многочленов. Рациональная дробь называется правильной, если степень числителя меньше степени знаменателя. В противном случае рациональная дробь называется неправильной. Всякую правильную рациональную дробь можно представить в виде суммы конечного числа простейших рациональных дробей первого и четвертого типов, причем каждому множителю знаменателя соответствует единственная дробь или сумма $k$ дробей:
	\begin{enumerate}
		\item $x - a \mapsto\frac{A}{x - a}$
		\item $(x - a)^k \mapsto\frac{A_k}{(x - a)^k} + \frac{A_{k-1}}{(x - a)^{k-1}} + \cdots + \frac{A_1}{x - a}$
		\item $x^2 + px + q\mapsto\frac{Ax + B}{x^2 + px + q}$
		\item $(x^2 + px + q)^k \mapsto\frac{A_kx + B_k}{(x^2 + px + q)^k} + \frac{A_{k-1}x + B_{k-1}x}{(x^2 + px + q)^{k-1}} + \cdots + \frac{A_1x + B_1}{x^2 + px + q}$
	\end{enumerate}
	Для интегрирования рациональной функции $R(x) = \frac{P_n(x)}{Q_m(x)}$, где $P_n(x)$ и $Q_m(x)$ ~--- полиномы $n$ и $m$ степеней соответственно нужно проверить:
	\begin{enumerate}
		\item Если $n \ge m\Rightarrow$ выделить целую часть,
		$$
			\frac{P_n(x)}{Q_m(x)} = S_{n - m}(x) + \frac{P_k(x)}{Q_m(x)}, 
		$$
		где $k < m$.
		\item Если $n < m\Rightarrow$ разложить на простейшие.
	\end{enumerate}
	\section{Разложение на простейшие}
	\textit{Простейшей дробью} называется рациональная дробь одного из следующих четырех типов:
	\begin{align}
		I. & \frac{A}{x-a} & II. & \frac{A}{(x-a)^n} & (n\ge 2).\\
		III. & \frac{Ax + B}{x^2 + px + q} & IV. & \frac{Ax + B}{(x^2 + px + q)^n} & (n \ge 2).
	\end{align}

	Здесь $A, B, a, p, q$ ~--- действительные числа, а трехчлен не имеет действительных корней, т.е. $\frac{p^2}{4}-q < 0$.

	Для нахождение коэффициентов разложения применяют метод неопределенных коэффициентов и метод частных значений.

	\textit{Метод частных значений} основан на том, что если два многочлена равны, то они равны при любых значениях аргумента.

	\textit{Метод неопределенных коэффициентов} основан на сравнении коэффициентов при одинаковых степенях $x$ левой и правой частей, т.е. если два многочлена равны, то, соответственно, равны их коэффициенты при одинаковых степенях $x$.
	\section{Интегрирование простейших дробей.}
	Простейшие дроби первого и второго типов интегрируются непосредственно с помощью основных правил интегрального исчисления.
	Интеграл от простейшей дроби третьего типа приводится к табличным интегралам путем выделения в числителе дифферницала знаменателя к сумме квадратов:
	\begin{align*}
		\int{\frac{(Ax+B)dx}{x^2 + px + q}} = 
		\begin{vmatrix}
			d(x^2 + px + q) = (2x + p)dx\\
			Ax + B = \frac{A}{2}(2x + p) + B - \frac{Ap}{2}
		\end{vmatrix}
		= \frac{A}{2} \int{\frac{d(x^2 + px + q)}{x^2 + px + q}} + \left(B - \frac{Ap}{2}\right)\int{\frac{dx}{x^2 + px + q}} = \\
		= \frac{A}{2}\ln{x^2 + px + q} + \left(B - \frac{Ap}{2}\right)\int{\frac{d\left(x + \frac{p}{2}\right)}{(x + \frac{p}{2})^2 + q - \frac{p^2}{4}}} =\\
		= \frac{A}{2}\ln{x^2 + px + q} + \left(B - \frac{Ap}{2}\right)\frac{1}{\sqrt{q - \frac{p^2}{4}}}\arctan\frac{2x + p}{\sqrt{4q - p^2}} + C.
	\end{align*}
	\section{Рекуррентная формула для 4 типа простейших}
	\begin{enumerate}
		\item Выделим в числителе производную квадратного трехчлена $x^2 + px + q,$ получим
		$$
			\frac{M}{2}\int\frac{(2x + p)dx}{(x^2 + px + q)^n} + \left(N - \frac{Mp}{2}\right)\int\frac{dx}{\left(\left(s + \frac{p}{2}\right)^2 + q - \frac{p^2}{4}\right)^n}.
		$$
		\item Первый интеграл $\int\frac{(2x + p)dx}{(x^2 + px + q)^n} = \frac{1}{(1 - n)(x^2 + px + q)^{n - 1}}$.
		\item Во втором интеграле введем подстановку $x + \frac{p}{2} = t$ и, обозначив $q - \frac{p^2}{4} = a^2$, приведем его к виду
		$$
			\int\frac{dt}{(t^2 + a^2)^n} = \frac{1}{a^2}\int\frac{t^2 + a^2 - t^2}{(t^2 + a^2)^n}dt = \frac{1}{a^2}\int\frac{dt}{(t^2 + a^2)^{n-1}}-\frac{1}{a^2}\int\frac{t^2dt}{(t^2 + a^2)^n}.
		$$
		\item Интегрирование $\int\frac{t^2dt}{(t^2 + a^2)^n}$ выполняется по частям. При $u = t, dv = \frac{tdt}{(t^2 + a^2)^n}$ будем иметь
		$$
			\int\frac{t^2dt}{(t^2 + a^2)^n} = \frac{1}{2}\frac{t}{(1 - n)(t^2 + a^2)^{n - 1}} - \frac{1}{2(1 - n)}\int\frac{dt}{(t^2 + a^2)^{n - 1}}
		$$
		\item Если обозначить $I_n = \int\frac{dt}{(t^2 + a^2)^n},$ то после простых преобразований получим формулу
		$$
			I_n = \frac{1}{2(n - 1)a^2}\cdot\frac{t}{(t^2 + a^2)^{n - 1}} + \frac{2n - 3}{2n - 2}\cdot\frac{1}{a^2}I_{n - 1}.
		$$
		\item Выражаем по такой же формуле $I_{n-1}$ через $I_{n - 1}, I_{n - 2}$ через $I_{n - 3}$ и т.д. Процесс продолжаем до тех пор, пока не дойдем до интеграла
		$$
			I_1 = \int\frac{dt}{t^2 + a^2} = \frac{1}{a}\arctg\frac{t}{a} + C.
		$$
		Формула для вычисления $I_n$ называется \textit{рекуррентной} (возвратной) формулой. Название объясняется тем, что для вычисления $I_n$ приходится возвращаться к $I_{n - 1},$ от него ~--- к $I_{n - 2}$ и т.д.
	\end{enumerate}
	\section{Определенный интеграл и его свойства}
	Пусть функция $f(x)$ определена на отрезке $[a, b]$ и $F(x)$ ~--- ее первообразная. Положим
	$$
		\int\limits_a^b f(x)dx = \left. F(x)\right|_a^b
	$$
	Выражение, стоящее в левой части формулы, называется \textit{определенным интегралом} от функции f по отрезку $[a, b]$, а числа $a$ и $b$ называются \textit{пределами интегрирования}.
	\begin{enumerate}
		\item Интеграл с одинаковыми пределами равен нулю: $\int\limits_a^a f(x)dx = 0$.
		$$
			\int\limits_a^b dx = b - a
		$$
		\item При перестановке пределов изменяется знак интеграла:
		$$
			\int\limits_a^b f(x) dx = -\int\limits_b^a f(x) dx.
		$$
		Постоянный множитель можно выносить за знак интеграла:
		$$
			\int\limits_a^b cf(x) dx = c\int\limits_a^b f(x) dx, c = const.
		$$
		Интеграл от суммы равен сумме интегралов от всех слагаемых:
		$$
			\int\limits_a^b (f_1(x)\pm f_2(x)\pm\ldots\pm f_n(x))dx = \int\limits_a^b f_1(x)dx\pm\int\limits_a^b f_2(x)dx\pm\ldots\pm\int\limits_a^b f_n(x)dx.
		$$
		\item Отрезок интегрирования можно разбивать на части:
		$$
			\int\limits_a^b f(x) dx = \int\limits_a^c f(x) dx + \int\limits_c^b f(x) dx.
		$$
		\item Если $f(x) \ge 0 \forall x\in [a, b], a < b, \text{то} \int\limits_a^b f(x) dx \ge 0.$
		\item Если $f(x) \ge\varphi (x) \forall x\in [a, b], a < b, \text{то} \int\limits_a^b f(x) dx \ge\int\limits_a^b\varphi (x) dx$.
		\item Если $m$ и $M$ ~--- соответственно наименьшее и наибольшее значения функции $f(x)$, непрерывной на отрезке $[a, b]$, то считая $a < b$, получим
		$$
			m(b - a)\le\int\limits_a^b f(x) dx \le M(b - a)
		$$
		\item Если функция $f(x)$ непрерывна на отрезке $[a, b]$, то существует такая точка $\xi\in [a, b]$, что
		$$
			\int\limits_a^b f(x) dx = f(\xi )(b - a).
		$$
	\end{enumerate}
	\section{Производная от интеграла с переменным верхним пределом}
	\paragraph{Теорема Барроу}
	Пусть $f\in \mathbb{R}(a, b)$ и непрерывна в $x_0\in (a, b)$. Тогда $F$ дифференцируема в этой точке и ее производная равна $F'(x_0) = f(x_0)$.
	$$
		F'(x_0) = \lim\limits_{\Delta x\to 0}\frac{F(x_0 + \Delta x) - F(x_0)}{\Delta x} = \lim\limits_{\Delta x\to 0}\frac{\int\limits_{x_0}^{x_0 + \Delta x}f(x) dt - \int\limits_a^{x_0} f(t) dt}{\Delta x} = \lim\limits_{\Delta x\to 0}\frac{\int_x^{\Delta x} f(t) dt}{\Delta x} = \lim\limits_{\Delta x\to 0}\frac{f(x_0) \Delta x}{\Delta x} = f(x_0)
	$$
	\section{Формула Ньютона-Лейбница}
	$$
		\int\limits_a^b f(x)dx = \left. F(x)\right|_a^b = F(b) - F(a)
	$$
	Формула выше называется \textit{формулой Ньютона-Лейбница}.
	\section{Площадь криволинейной трапеции}
	Пусть на отрезке $[a, b], a < b$ задана непрерывная неотрицательная функция $f(x)$.
	Тогда площадь $S$ фигуры, ограниченной осью абсцисс, прямыми $x = a, x = b$ и графиком функции $y = f(x)$ вычисляется по формуле.
	$$
		S = \int\limits_a^b f(x) dx.
	$$
	Эта фигура называется \textit{криволинейной трапецией}.
	Если на отрезке $[a, b]$ заданы две непрерывные функции $f(x)$ и $g(x)$, причем $f(x) < g(x)$, при $x \in [a, b]$, то площадь $S$ фигуры, ограниченной прямыми $x = a, x = b$ и линиями $y = f(x), y = g(x)$ вычисляется по формуле
	$$
		S = \int\limits^b_a [f(x) - g(x)] dx
	$$
	Площадь $S$ криволинейного сектора, ограниченного лучами $\varphi = \alpha, \varphi = \beta (\alpha < \beta)$ и кривой, заданной уравнением в полярных координатах $\rho = g(\varphi )$, находится из равенства
	$$
		S = \frac{1}{2}\int\limits^\beta_\alpha \rho ^2 d\varphi = \frac{1}{2}\int\limits^\beta_\alpha (g(\varphi ))^2 d\varphi
	$$
\end{document}