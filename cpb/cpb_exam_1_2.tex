\documentclass[11pt]{article}

\usepackage[T2A]{fontenc}
\usepackage[utf8x]{inputenc}

\usepackage[russian, english]{babel}

\usepackage{vmargin}
\setpapersize{A4}
\setmarginsrb{1cm}{1cm}{1cm}{1cm}{0pt}{0pt}{0mm}{13mm}

\usepackage{amssymb, amsmath}

\usepackage{graphicx}

\sloppy

\begin{document}
	\section{Представление чисел с фиксированной точкой. Прямой, обратный и дополнительный код. Формирование битовых признаков переноса, переполнения, отрицательного результата, нуля}
	\paragraph{Представление чисел с фиксированной точкой.}
	В работе. Мне нужно перезагрузиться в другую систему, дабы найти там материалы.
	\paragraph{Прямой, обратный и дополнительный код.}
	Есть три представления чисел с фиксированной точкой в ЭВМ:
	\begin{itemize}
		\item Прямой ~--- модуль числа представляется, что для положительных, что и для отрицательных одинаково. Под знак числа отводится старший разряд. Область значений таких чисел симметрична, но ограничена тем, что у нас есть два представления нуля (положительный и отрицательный). Для N-разрядного целого двоичного числа область значений ~--- $[-2^{N-1}; 2^{N-1}]$;
		\item Обратный код ~--- представление положительных чисел сходно с прямым. Для представления отрицательных чисел используется инверсия положительного числа того же модуля. Для N-разрядных целых двоичных чисел область значений ~--- $[-2^{N-1}+1;2^{N-1}-1]$. Недостаток с двойным кодированием нуля также присутствует, но область значений также симметрична.
		\item Дополнительный код ~--- представление положительных чисел также как и в прямом коде. Код отрицательных чисел формируется посредством инверсии всех разрядов положительного числа и прибавлении единицы. Область значений уже не симметрична (отрицательных чисел на одно больше, если ноль считать ни тем не другим), но исправляется проблема с двойным кодированием нуля. Для N-разрядной сетки двоичных целых чисел область значений ~---$[-2^{N-1};2^{N-1}-1]$. Обоснование формулы для N-разрядных:
		\begin{align*}
			\text{Отрицательное число это:} & M' = 2^N - M,
			\text{где } & M\ge 0\\
			\text{Прибавим и вычтем из формулы 1:} & M' = 2^N - 1 + 1 - M =\\
			= ((2^N - 1) - M) + 1 &= \bar M + 1
		\end{align*}
	\end{itemize}
	\paragraph{Формирование битовых признаков результата.}
	Регистр флагов ~--- регистр процессора (FLAGS), отражающий текущее состояние процессора. Регистр флагов содержит группу флагов состояния (арифметические флаги) и флаги управления. В БЭВМ регистром состояния является регистр PS (Program State) в его младших 4 разрядах хранятся флаги NZVC.
	\begin{description}
		\item[Перенос (C)] CF (Carry Flag) ~--- арифметический флаг переноса, в нем фиксируется перенос из старшего разряда при сложении и заем в старший разряд при вычитании. При умножении CF показывает возможность (=0) и невозможность (=1) представления произведения в том же формате, что и операндов. Флаг переноса является индикатором ошибки переполнения в беззнаковой арифметике. Используется в командах ветвления (условных переходах) для беззнаковой арифметики. В БЭВМ устанавливается по результату только тогда, когда открыт вентиль SETC, для этого используются 3 сигнала (выходящие из АЛУ): $С_O$ (Carry old ~--- флаг переноса до исполнения команды), $C_N$ (Carry new ~--- новый флаг, сформировавшийся после исполнения команды), $C_{14}$ (Carry 14 ~--- перенос в 14 разряд). На вход коммутатора пропускаются только 2 бита, связанные с переносом: $C_N$ (Либо как $C_N$ с АЛУ, либо как $C_O$ в зависимости от установки вентилей) и $C_{14}$, использующийся для выставления флага переполнения.
		\item[Переполнение (V)] OF (Overflow Flag) ~--- флаг переполнения. Устанавливается в командах сложения и вычитания, если результат не помещается в формате, при этом и операнды и результат интерпретируются как знаковые числа. Аппаратно он формируется совпадением переносов из двух старших разрядов при сложении и заемов в два старших разряда при вычитании (если они совпадают, то флаг равен 0). Переполнение фиксируется 3 способами:
		\begin{itemize}
			\item сравнение знаков операндов и суммы: если знак суммы отличается от знаков операндов, то фиксируется переполнение;
			\item сравнение переносов из двух старших разрядов: если они не совпадают, то фиксируется переполнение;
			\item использование модифицированного знака (под знак отводится два разряда, второй дублирует знак).
		\end{itemize}
		В БЭВМ флаг переполнения помимо того, что он выставляется лишь при открытом вентиле SETV, использует биты $C_N$ и $C_{14}$ по следующей формуле: $V = C_N \oplus C_{14}$.
		\item[Отрицательный результат (N)] SF (Sign Flag) ~--- флаг знака, в котором копируется старший разряд результата. В БЭВМ копируется именно 15 бит результата при открытом вентиле STNZ.
		\item[Флаг нуля (Z)] ZF (Zero Flag) ~--- флаг нуля, устанавливается при нулевом значении результата, в противном случае сбрасывается. В БЭВМ устанавливается с помощью 16 входового элемента ИЛИ-НЕ (NOR) при открытом вентиле STNZ.
	\end{description}
	\section{Базовые элементы вычислительной техники: ячейки, регистры, шины, вентили, тактовые генераторы, логические схемы, триггеры, счетчики, сумматоры}
	\paragraph{Ячейки}
	Для хранения информации в ЭВМ используются ячейки памяти двух видов:
	\begin{enumerate}
		\item SRAM ~--- Static Random Access Memory (Используется в основном в ПЗУ и его видах). Работает на 6 транзисторах за счет положительно-обратной связи. Не требует постоянной подзарядки, данные хранятся и без нее.
		\item DRAM ~--- Dynamic Random Access Memory (Используется в основном в ОЗУ). Требует лишь один транзистор и конденсатор (можно без если транзистор сам имеет паразитную емкость). Требует постоянной подзарядки из-за разрядки конденсатора, иначе данные теряются.
	\end{enumerate}
	\paragraph{Регистры}
	\paragraph{Шины}
	Набор проводов для передачи информации между компонентами логических схем. Имеет разрядность передаваемой информации, указанную обычно косой чертой на шине с числом разрядов рядом. На своих концах требует установку заглушек, для предотвращения отражения сигнала.
	\paragraph{Вентили}
	Один из компонентов логических схем, предназначенный для пропускания или задержки сигнала. Имеет два входа (входной сигнал и управляющий) и один выход (выходной сигнал). По сути работает как элемент И при положительном кодировании. При отсутствии управляющего сигнала (0), не пропускает сигнал на выход (0), при наличии управляющего сигнала (1) ~--- пропускает вход (1/0).
	\paragraph{Тактовые генераторы}
	\paragraph{Защелки}
	\paragraph{Триггеры}
	\paragraph{Счетчики}
	\paragraph{Сумматоры}
\end{document}
